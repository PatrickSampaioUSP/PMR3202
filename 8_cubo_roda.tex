\section{Cubo-Eixo}

Transmissão do movimemnto do eixo ao cubo, portanto transferência de momento torsor

\subsection{Fixação Cubo-Eixo por Atrito}

Fixação através de um ajuste forçado, portanto utilizando somente o atrito para fixar.

\begin{itemize}
	\item Limite do torque para que não haja deformação plastica do eixo
	\item Não necessita usinagem do eixo
	\item Não adiciona uniformidades nas peças, portanto não há concentradores de tensão
	\item Baixo custo
	\item Dano na desmontagem
\end{itemize}

\subsection{Assento Cônico}

Através de uma parte cônica do eixo e uma porca na parte cilindrica, fixa-se o eixo ao cubo pela força axial da porta, e há transmissão de momento torsor através do atrito entre o assento cônico e a parte conica do eixo

\includegraphics[width=4cm]{C:/Users/patri/Downloads/Poli/lateco/carrinhos_teoria/images/conico.png}

\begin{itemize}
	\item Limite elástico dos materiais
	\item Não causa danos na desmontagem
	\item Necessidade de usinagem
	\item Gera concentração de tensão
	\item Alto custo
\end{itemize}

\subsection{Bucha Cônica}

Coloca-se uma bucha que vai fixa ao eixo, de maneira que se possa utilizar um sistema de fixação em qualquer parte ao longo do eixo. Funciona como a fixação do Assento Cônico, porcas prendem por tensões axiais a bucha, que irá transmitir os esforços do eixo para o cubo pelo atrito.

\includegraphics[width=4cm]{C:/Users/patri/Downloads/Poli/lateco/carrinhos_teoria/images/assento_conico_bucha.png}

\begin{itemize}
	\item Limite elástico dos materiais
	\item Não causa danos na desmontagem
	\item Necessidade de usinagem
	\item Gera concentração de tensão
	\item Ajuste axial e angular eixo-cubo
	\item Alto custo
\end{itemize}

\subsection{Cubo Bipartido}

Cubo é fixado através de elementos de fixação em ambos lados. O controle da força de compressão do cubo ao eixo é dada pelo torque que é aplicado nos elementos de fixação. O peso em ambos lados deve ser igual para que não haja desbalanceamento.

\includegraphics[width=4cm]{C:/Users/patri/Downloads/Poli/lateco/carrinhos_teoria/images/cubo_bipartido.png}

\begin{itemize}
	\item Há problemas para limitar a força no limite elástico no material
	\item Ajuste axial e angular
	\item Sem concentração de tensão
	\item Não há danos na desmontagem
	\item Alto custo
\end{itemize}

\subsection{Fixação Cubo-Eixo por Adesão}

União em que utiliza-se de um adesivo para fixar o sistema Cubo-Eixo, havendo limitação do torque torsor pelo limite de cisalhamento do mesmo.

\includegraphics[width=4cm]{C:/Users/patri/Downloads/Poli/lateco/carrinhos_teoria/images/fixacao_adesivo.png}

\textbf{Área da união}: $S_a = l_a \pi d$
\textbf{Força cisalhamento no adesivo}: $F_t = 2\frac{M_t}{d}$
\textbf{Tensão de cisalhamento}: $\frac{F_t}{S_a}$

Sem ruptura da união sem $t \leq t_a$. Portanto as variáveis em que há possibilidade de se alterar é a área de união e a tensão máxima de cisalhamento do adesivo. 

$$I_a \geq \frac{2M_t}{(d^2t_a\pi)} $$

$I_a$ é o comprimento axial.

\begin{itemize}
	\item Alto custo
\end{itemize}

\subsection{Fixação por Travamento}

Travamento através de elementos mecânicos, num primeiro caso, utiliza-se um pino transversal para realizar o travamento do sistema cubo-eixo, permitindo maiores torques transmitidos.

\includegraphics[width=4cm]{C:/Users/patri/Downloads/Poli/lateco/carrinhos_teoria/images/fixacao_travamento.png}

Ajuste aderente do pino ao sistema cubo-eixo. O pino pode fallhar por cisalhamento.

\begin{itemize}
	\item Conjugado no eixo: $M_t = F_t * \frac{d}{2}$
	\item Diametro do eixo: $d$
	\item Tensão admissível ao cisalhamento no pino: $t_a$
	\item Diametro do pino: $d_p$
\end{itemize}

Formula para dimensionamento do pino:
$\boxed{d_p \geq 2\sqrt{\frac{M_t}{\pi d t_a}} }$

\begin{itemize}
	\item Necessidade de usinagem tanto do cubo tanto do eixo
	\item Elemento de fixação extra
	\item Torque transmitidos são superiores
	\item Facil desmontagem
	\item Sem ajuste angular nem axial
	\item Gera descontinuidade na peça, portanto concentração de tensão
\end{itemize}

\subsection{Chavetas ou perfis entalhados}

Eixo com chaveta que encaixa num rasgo do cubo, transmitindo o movimento pela fixação mecânica da chaveta. Ou pode-se fazer um perfil entalhado que encaixa no perfil do cubo

\includegraphics[width=4cm]{C:/Users/patri/Downloads/Poli/lateco/carrinhos_teoria/images/fixacao_chaveta.png}

\includegraphics[width=4cm]{C:/Users/patri/Downloads/Poli/lateco/carrinhos_teoria/images/eixo_entalhado.png}

\begin{namedtheorem}[Falha por cisalhamento em chaveta] 

\begin{itemize}
	\item Área da União: $S_t = bL_t$
	\item Força de cisalhamento da união: $F_t = 2\frac{M_t}{d}$
	\item Tensão de cisalhamento uniforme: $t = \frac{F_t}{S_t}$
\end{itemize}

Para não haver deformação plastica:

$\boxed{L_t \geq \frac{2M_t}{d t_a b} }$

\end{namedtheorem}

\begin{namedtheorem}[Falha por esmagamento em chaveta]

\begin{itemize}
	\item Área da União: $S_c = \frac{h}{2}L_c$
	\item Força de compressão da união: $F_c = 2\frac{M_t}{d}$
	\item Tensão de compressão uniforme: $\sigma = \frac{F_c}{S_c}$
\end{itemize}

Para não haver deformação plastica:

$\boxed{L_c \geq \frac{4M_t}{d \sigma_a h} }$

\end{namedtheorem}

\begin{namedtheorem}[Falha por cisalhamento no entalho]

\begin{itemize}
	\item $N$ é o numero de dentes do entalho
	\item $\eta$ é o coeficiente de correção de carga, geralmente $\eta = 1.25$
\end{itemize}

Para não haver deformação plastica:

$\boxed{L_{et} \geq \frac{2M_t\eta}{d t_a b N} }$

\begin{namedtheorem}

\begin{namedtheorem}[Falha por esmagamento no entalho]

\begin{itemize}
	\item $N$ é o numero de dentes do entalho
	\item $\eta$ é o coeficiente de correção de carga, geralmente $\eta = 1.25$
\end{itemize}

Para não haver deformação plastica:

$\boxed{L_{ec} \geq \frac{2M_t\eta}{d \sigma_a h_e N} }$

\begin{namedtheorem}

Norma técnica \textbf{DIN 6885} mapea, em função do diametro, outros parametros para o projeto, como largura e altura da chaveta plana no cubo e no eixo.

Caracteristicas

\begin{itemize}
	\item Conjugados moderados à elevados
	\item Desmonta facilmente
	\item Não há ajuste axial nem angular
	\item Usinagem extensa, tanto do cubo quanto do eixo, além de um novo elemento no caso da chaveta
	\item Há adição de descontinuidade, portanto havendo concentração de tensão
	\item Custo elevado
\end{itemize}



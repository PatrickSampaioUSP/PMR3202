\section{Elementos Rosqueados}

Forma de união que consiste na utilização do atrito para prender duas peças, onde há um parafuso que se encaixa em um furo rosqueado. Sistema amplamente utilizado e muito versatíl.

\begin{namedtheorem}[Rosca]
	A definição de Rosca é o conjunto de filetes de um elemento.
	\begin{center}
	\includegraphics[width=4cm]{C:/Users/patri/Downloads/Poli/lateco/carrinhos_teoria/images/filete.png}\\
	\end{center}
	Onde este filete há algumas características importantes, como o tamanho, o espaçamento e o avanço.
\end{namedtheorem}

\begin{namedtheorem}[Passo]
	Distância entre mesmos pontos de filetes consecutivos
\end{namedtheorem}

\begin{namedtheorem}[Avanço]
	Distância percorrida axialmente por uma rotação completa do parafuso
\end{namedtheorem}

\begin{namedtheorem}[Angulo de Avanço]
	Angulo entre o filete e o plano perpendiular à normal do eixo do parafuso
\end{namedtheorem}

\begin{namedtheorem}[Avanço]
	$\textit{Avanço} = \textit{Número de entradas}*\textit{Passo}$
\end{namedtheorem}

\begin{center}
\includegraphics[width=8cm]{C:/Users/patri/Downloads/Poli/lateco/carrinhos_teoria/images/passovsavanco.png}\\
\end{center}

\subsection{Nomenclatura}

\begin{itemize}
	\item \textbf{\textcolor{red}{M10}}: Rosca métrica de diâmetro nominal 10mm e passo normal
	\item \textbf{\textcolor{red}{M10x0,75}}: Diametro nominal 10mm e passo fino de 0,75mm
	\item \textbf{\textcolor{red}{M10x25}}: Diametro nominal 10mm, passo normal com 25mm de comprimento
	\item \textbf{\textcolor{red}{M10x0,75x25}}: Diametro nominal 10mm, passo fino de 0,75mm e 25mm de comprimento
\end{itemize}

Portanto temos que:

$\boxed{Nomenclatura = Diametro_{nominal}XPassoXComprimento}$

\subsection{Passo Fino X Passo Normal}

Aspectos da rosca de passo fino

\begin{itemize}
	\item Maior aperto devido à maior área de contato
	\item Maior precisão no ajuste
	\item Menor avanço por rotação
\end{itemize}

\begin{center}
\includegraphics[width=8cm]{C:/Users/patri/Downloads/Poli/lateco/carrinhos_teoria/images/passo_fino_vs_passo_normal.png}\\
\end{center}

\subsection{Uniões Parafusadas}

As dimensões de um parafuso, para dado um diametro podem ser consultadas na tabela nos apêndices.

\begin{namedtheorem}[Parafusos Passantes]
	Parafuso atravesa as duas peças à serem unidas, sendo necessário adição de porcas e arruelas para prender as duas peças.
	\begin{center}
	\includegraphics[width=6cm]{C:/Users/patri/Downloads/Poli/lateco/carrinhos_teoria/images/parafuso_passante.png}\\
	\end{center}
\end{namedtheorem}

\begin{namedtheorem}[Parafusos Não-Passantes]
	Uniões não passantes são uniões que envolvem a furação das peças a serem unidas e fazer um furo roscado para rosquear o parafuso
	\begin{center}
	\includegraphics[width=6cm]{C:/Users/patri/Downloads/Poli/lateco/carrinhos_teoria/images/furo_nao_passante.png}\\
	\end{center}
	Esta união é mais complicada de ser feita, pois envolve fazer um furo menor que o diametro do parafuso e a aplicação de um conjunt de machos para fazer a rosca no furo
\end{namedtheorem}

\subsection{Cargas}

\begin{itemize}
	\item \textbf{Tração}: Carga de trabalho dos parafusos rosqueados
	\item \textbf{Cisalhamento}: Carga de trabalho dos parafusos de corpo liso
\end{itemize}

\subsection{Modos de Falha}

A maioria dos modos de falhas estão associados à

\begin{itemize}
	\item Dimensionamento do diametro do parafuso com baixo fator de segurança
	\item Falha devido à concetração de tensão gerada pelos furos
\end{itemize}

\begin{center}
\includegraphics[width=6cm]{C:/Users/patri/Downloads/Poli/lateco/carrinhos_teoria/images/bolt_failure.jpg}
\end{center}

\subsection{Vantagens}

\begin{itemize}
	\item Totalmente Desmontável
	\item Peças amplamente disponíveis e consolidadas
	\item Não há necessidade de mão de obra qualificada
	\item Ajuste do aperto entre as peças
\end{itemize}

\subsection{Desvantagens}

\begin{itemize}
	\item Enfraquecimento das peças pela introdução de descontinuidade na peça pelo furo
	\item Adição de custos pelos elementos rosqueados
	\item Concentração de tensão nos ultimos filetes da rosca
	\item Susceptível a vibrações
\end{itemize}



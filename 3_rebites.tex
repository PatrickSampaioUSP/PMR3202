\section{União por Rebites}

\subsection{Características Gerais}

Dispositivo feito para junção de duas peças atraveś da inserção de um pino. O principal regime de esforço ao qual a junção será exposta, é um \textbf{regime de cisalhamento}

\begin{namedtheorem}[Cálculo dimensional de união rebitada]
  \[ \tau_{rebite} = \frac{F_r}{\frac{\pi d_{ri}^2}{4}} \leq \tau_{admissivel} \]
\end{namedtheorem}

\begin{itemize}
	\item O material do rebite geral é alumínio
	\item Uniões rebitadas devem ser feitas com folga entre o furo e o eixo do rebite.
\end{itemize}

\subsection{Tipos de rebites}

\begin{namedtheorem}[Rebite Pop]
  Fixação através da deformação realizada ao puxar a cabeça inferior do rebite, nessa deformação a força irá quebrar o corpo do rebite, fixando-o. Um de seus problemas e ele não ser maciço

\begin{center}
	\includegraphics[width=4cm]{C:/Users/patri/Downloads/Poli/lateco/carrinhos_teoria/images/rebite_repuxo.png}  
\end{center}

\end{namedtheorem}

\subsection{Tipo de União}

\begin{itemize}
	\item \textbf{A Frio}: Utilizado em rebites de pequeno diametro, requer pouca ferramenta.
	\item \textbf{A Quente}: Utilizado em rebites de grandes diametros, adiciona uma tensão de compressão extra devido ao resfriamento que aumenta a \textit{resistência ao cisalhamento} da união.
\end{itemize}

\subsection{Falhas nos Rebites}

\begin{center}
\includegraphics[width=8cm]{C:/Users/patri/Downloads/Poli/lateco/carrinhos_teoria/images/rebite_flaws.png}  
\end{center}

\begin{itemize}
	\item \textbf{Corte de Rebite}: União rebitada foi exposta a uma carga superior à dimensionada no projeto.\\ \textbf{Como Evitar?} Aumento do Fator de Segurança do projeto.
\end{itemize}

\subsection{Vantagens da União Rebitada}

\begin{itemize}
	\item Preço
	\item Alta resistência da união
	\item Não é necessário mão de obra para execuçao da união.
	\item Descontinuidade estrutural gerada pelo furo, impede a propagação de trincas entre as peças.
\end{itemize}

\subsection{Desvantagens da União Rebitada}

\begin{itemize}
	\item Peso da estrutura mais elevado devida a adição de elementos
	\item Necessidade de furação das partes
	\item Introdução de pontos de aumento de tensão.
\end{itemize}
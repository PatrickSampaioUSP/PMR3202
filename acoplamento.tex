\section{Fixação Eixo-Eixo}

Elementos de máquina cujo objetivo é a transmissão de momento torsor.

\subsection{Acoplamento}

\begin{itemize}
	\item Montagem e desmontagem para manutenção do equipamento
	\item Compensar desalinhamento entre os eixos
	\item Absorver ou isolar as vibrações entre as máquinas
\end{itemize}

\subsection{Tipos de Acoplamentos}

\begin{namedtheorem}[Rígidos]
	\begin{itemize}
		\item Não permitem desalinhamento
		\item Torsionalmente rígidos, resistencia a torsão muito alta, baixa chance de deformação de torsão elástica
		\item Transmitem choques e vibração
		\item Baixo Custo
		\item Não exigem manutenção
	\end{itemize}
	
	\begin{center}
\includegraphics[width=4cm]{C:/Users/patri/Downloads/Poli/lateco/carrinhos_teoria/images/acoplamento_rigido.jpg}
	\end{center}
\end{namedtheorem}

\begin{namedtheorem}[Possíveis Desalinhamentos]
	\begin{center}
	\includegraphics[width=4cm]{C:/Users/patri/Downloads/Poli/lateco/carrinhos_teoria/images/acoplamento_rigido.jpg}
	\end{center}
	
	\begin{enumerate}
		\item Desalinhamento Axial
		\item Desalinhamento Radial
		\item Desalinhamento Angular
	\end{enumerate}
	
	Em campo, ocorrem mistura destes desalinhamentos. Ter uma acoplamento que exige alinhamento perfeito significa aumento do custo de montagem.
\end{namedtheorem}

\begin{namedtheorem}[Elásticos]
	Características variam conforme o tipo de acoplamento flexível utilizado.
	\begin{itemize}
		\item Pelo fato de serem elástico não possuem sincronismo entre eixo motor e eixo movido
		\item Podem ser torcionalmente rígidos ou flexíveis
		\item Absorvem/isolam choques e vibrações mecânicas
		\item Exigem lubrificação manutenção
		\item Elemento mais caro
	\end{itemize}
	
	\begin{center}
	\includegraphics[width=4cm]{C:/Users/patri/Downloads/Poli/lateco/carrinhos_teoria/images/acoplamento_elastico_rigido.png}
	\end{center}	
	
	O funcionamento deste acoplamento divide-se em algumas etapas
	
	\begin{itemize}
		\item Momento torsor do eixo motor é transmitido ao cubo de eixo interno do acoplamento
		\item Cubo de eixo possui perfil entalhado transmite torque à carcaça externa do acoplamento
		\item Carcaça externa transmite torque ao cubo de eixo com perfil entalhado do eixo movido
		\item Cubo de eixo transmite torque ao eixo movido
	\end{itemize}
	
	Possui pequena tolerância a desalinhamento através do travamento do entalhe, cubo desliza num canal interno da carcaça interna.
	
	Qualidades:
	\begin{itemize}
		\item Grande flexibilidade angular
		\item Sincronismo entre eixo motor e eixo movido
		\item Permite desalinhamento entre os eixos
	\end{itemize}
	
	Desvantagens:
	\begin{itemize}
		\item Exigem Lubrificação, graxa
		\item Alto custo devida ao preço de usinagem
		\item Absorvem pouca vibração e choque mecânico
	\end{itemize}
	
	\begin{center}
	\includegraphics[width=4cm]{C:/Users/patri/Downloads/Poli/lateco/carrinhos_teoria/images/acoplamento_lamela.png}
	\end{center}		
	
	Pinos ligam as partes através de furos conicos que permitem a compensação à desalinhamentos. Possuem molas para resistencia mecância
	
	\textbf{Totalmente Flexíveis}
	
	\begin{center}
	\includegraphics[width=4cm]{C:/Users/patri/Downloads/Poli/lateco/carrinhos_teoria/images/acoplamento_elastico.jpg}
	\end{center}		
	
	Qualidades
	\begin{itemize}
		\item Maior flexibilidade de desalinhamento
		\item Absorção/isolamento de vibrações/choques
	\end{itemize}
	
	Defeitos
	\begin{itemize}
		\item Menor rigidez
		\item Problemas de sincronismo entre os eixos
		\item Manutenção/lubrificação
		\item Custo Elevado
	\end{itemize}
	
\end{namedtheorem}

\begin{namedtheorem}[Para grandes movimentações angulares]

Consegue ligar eixos que estão em grandes distâncias entre si, e que através de uma \textbf{cruzeta} é possível grandes movimentações angulares. Alta capacidade de transmissão de torque.

	\begin{center}
	\includegraphics[width=4cm]{C:/Users/patri/Downloads/Poli/lateco/carrinhos_teoria/images/eixo_cardan.png}
	\end{center}		
	
	Características positivas:
	
	\begin{enumerate}
		\item Grande transmissão de torque
		\item Grande flexibilidade angular
		\item Bom sincronismo entre os eixos, materiais metálicos
	\end{enumerate}
	
	Características negativas
	
	\begin{enumerate}
		\item Necessidade de lubrificação 
		\item Não absorvem/isolam esforços mecânicos
		\item São de custo elevado
	\end{enumerate}

\end{namedtheorem}

\begin{namedtheorem}[Junta Homocinética]

	\begin{center}
	\includegraphics[width=4cm]{C:/Users/patri/Downloads/Poli/lateco/carrinhos_teoria/images/junta_homocinetica.jpg}
	\end{center}		

\end{namedtheorem}

\subsection{Critérios de Seleção}

\begin{itemize}
	\item Torque da máquina
	\item Regime de operação: contínuo, intermitente...
	\item Rotação máxima de operação
	\item Tipos de desalinhamentos
\end{itemize}











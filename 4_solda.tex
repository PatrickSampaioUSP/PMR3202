\section{Solda}

União que consiste no derretimento de materiais para a realização da união, podendo ou não haver metal adicionado. Neste tipo de união fatores como tensões residuais, trincas são importantes.

\subsection{Fusão do material de enchimento}

Neste tipo de solda, o material de enchimento possui uma temperatura de fusão inferior ao material base.

\begin{itemize}
	\item \textbf{Brazagem(brazing)}: Material de enchimento apresenta $T_{fusão} > 450ºC$.
	\item \textbf{Solda Fraca(soldering)}: Material de enchimento apresenta $T_{fusão} < 450ºC$.
\end{itemize}

Geralmente o material de enchimento para este tipo de solda é o estanho ou prata.

\subsection{Fusão do material de enchimento e do base}

\begin{itemize}
	\item \textbf{Soldagem à gas}: Utiliza o calor de uma combustão à gas como fonte de calor
	\item \textbf{Soldagem a arco-elétrico}:  Proteção é feita pelo revestimento do eletrodo, que ao se gaseficar irá proteger a região fundida para não oxidar. Há formação de escória, uma camadaprotetora de sais que não se mistura com a poça fundida. É um dos mais utilizados na industria, cerca de 50pct das soldagens de manu-tenção utilizam este processo.
	\item \textbf{MIG(Metal Inertial Gás)}: Utiliza um arco elétrico com soma de um gás inerte que sai da tocha protegendo a poça de oxidação. Altamente Automatizável.
	\item \textbf{TIG(Tungstein Inertial Gás)}: Mesma lógica do MIG, mudando apenas a adição de material de adição por fora, eletrodo de tungstênio não é consumivel.
\end{itemize}

\subsection{Fusão do Material base}
	Apenas material que compõem as partes a serem unidas são utilizados. A fonte de calor para este processo é alguma fonte altamente concentrada, como \textit{feixe de elétrons} ou \textit{raio laser}. Altamente automatizável e possui como vantagem não adicionar peso à estrutura.
	
\subsection{Problemas da soldagem}

\begin{center}
\includegraphics[width=8cm]{C:/Users/patri/Downloads/Poli/lateco/carrinhos_teoria/images/welding_deffect.png}  
\end{center}

Muitos dos problemas associados a soldagem estão relacionados à habilidade de operação de quem faz a solda, uma mesma técnica aplicada com diferente qualidade operacional resulta em junções extremamente diferentes em termos de qualidade nas característica. Isto posto, elencaremos alguns possíveis problemas:

\begin{itemize}
	\item \textbf{Trincas}: Defeito grave, sobretudo trincas internas, assoaciado a técnica, geralmente é gerado por movimentos bruscos durante a operação de soldagem
	\item \textbf{Impurezas}: Defeito associado a junção de elementos de escória à junção, problema associado tanto à falta de limpeza do equipamento quanto a habilidade do operador. Há técnicas para remover escória, por exemplo: A cada cordão aplicado, passar uma escova de aço na junção para remoção das impurezas
	\item \textbf{Porosidade}: Defeito associado a soldagem muito rápida, causa enfraquecimento mecânico da junção
	\item \textbf{Cavidades}
	\item \textbf{Fusão Imcompleta}
\end{itemize}

\subsection{Vantagens da soldagem}
\begin{itemize}
	\item Menor peso adicionado em relação a outros métodos de união.
	\item Não há limitação da espessura das partes à serem unidas.
	\item Automatizável em certos casos
	\item Aumento da eficiencia mecânica?
	\item Menor tempo de fabricação?
\end{itemize}

\subsection{Desvantegens da soldagem}
\begin{itemize}
	\item Habilidade do operador possui grande influência nas características da união
	\item Necessita de controle de qualidade rigoroso
	\item Tensões térmicas devido ao resfriamento não uniforme podem reduzir a eficiência mecânica da união
\end{itemize}


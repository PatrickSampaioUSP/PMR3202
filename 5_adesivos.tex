\section{Adesivos}

Fixação por adesivos consiste na união de peças através de interações de \textit{Van de Waals}, portanto forças intermoleculares. Neste processo há \textbf{tempo de cure} da união, portanto um fator a ficar atento, além de que exige-se que as superfícies a serem adesivadas sejam preparadas para havaer aderencia e durabilidade da cola.

\subsection{Processos de Cura}

Algumas maneiras de se realizar a cura do processo.

\begin{itemize}
	\item Reação Anaeróbica, isto é, privação de $O_2$
	\item Exposição ao Calor 
	\item Luz UV
\end{itemize}

\subsection{Preparação da Superfície}

Algumas medidas precisam ser tomadas para a preparação de uma superfície para adesivagem

\begin{itemize}
	\item Rugosidade $R_a < 3,5 \mu m$
	\item Remoção de partículas soltas
	\item Desengraxamento
	\item Ataque químico para formação de cavidades para aumentar aderencia
\end{itemize}

\subsection{Carga}

A união por adesivagem trabalha com \textbf{tensões de cisalhamento}.\\
A Equação de tensão é definida como $$\sigma = \frac{F}{A}$$.\\

Portanto é interessante a maior área possível, de forma que a tensão irá se distribuir ao longo da área reduzindo a chance de falha.\\
A escolha do tipo de cola deve levar em consideração a sua tensão de \textbf{cisalhamento máxima} e as temperaturas de operação.

\begin{center}
\includegraphics[width=8cm]{C:/Users/patri/Downloads/Poli/lateco/carrinhos_teoria/images/adesivagem.png}
\end{center}

\subsection{Modos de Falha}

A maioria dos modos de falha estão associado a 

\begin{enumerate}
	\item Carga Aplicada superior a tensão de cisalhamento que a junção aguenta
	\item Carga Aplicada superior a tensão de cisalhamento que as peças unidas aguentam
	\item Também há modos de falha associados ao envelhecimento da junção causado pelo calor
\end{enumerate}

\begin{center}
\includegraphics[width=8cm]{C:/Users/patri/Downloads/Poli/lateco/carrinhos_teoria/images/glue_bonding_failure.jpg}
\end{center}

\subsection{Vantagens}

\begin{itemize}
	\item Não adiciona peso à estrutura
	\item Baixo custo
	\item Automatizável
	\item Não há alteração estrutural nas peças
	\item Distribuição uniforme de tensões
\end{itemize}

\subsection{Desvantagens}

\begin{itemize}
	\item Necessidade de preparação da superfície
	\item Processo de cura adicionado à linha de fabricação da estrutura
	\item Não resiste a temperatura, ocorre envelhecimento pelo calor
\end{itemize}
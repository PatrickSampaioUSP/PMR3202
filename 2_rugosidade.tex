\section{Rugosidade}

Grandeza que mede o quão irregular é uma supefície.\\
Usualmente a \textbf{rugosidade} é uma medida \textbf{estatística} do perfil de uma peça, isto é, trabalha-se com uma amostragem do perfil, e através de métricas, como \textbf{Rugosidade Média} e \textbf{Rugosidade Máxima} tenta-se classificar tal perfil.\\ As técnicas de medida de rugosidade consistem em alpapar a superfície, de sorte que, a primeira e a última partição de médida são desconsideradas, considerando que há aumento do erro de medida devido à aceleração/desaceleração do medidor.
\begin{center}
\includegraphics[width=4cm]{C:/Users/patri/Downloads/Poli/lateco/carrinhos_teoria/images/carrinhos_2_rugo_a.png}
\end{center}

\begin{itemize}
	\item $l_m$: comprimento total de avaliação
	\item $l_t$: comprimento total
	\item $l_e$: comprimento de amostragem
\end{itemize}

\textbf{Linha média}: Área acima da reta e área abaixo da linha do perfil possuem a mesma área.

\begin{namedtheorem}[Rugosidade Média(Ra)]
  Média aritmética da ordenada do perfil de uma superfície. As ordenadas são feitas em módulo
\end{namedtheorem}

$$ R_a = \frac{1}{l_m} \int_{0}^{l_m} \norm{y} dx$$

Qualidade
\begin{itemize}
	\item Equipamento altamente disponível
\end{itemize}

Defeitos
\begin{itemize}
	\item Não descreve o perfil de maneira representativa pelo fato de ser a média, não detecta mudanças bruscas.
	\item
\end{itemize}

\begin{center}
	\includegraphics[width=4cm]{C:/Users/patri/Downloads/Poli/lateco/carrinhos_teoria/images/carrinhos_2_rugo_a.png}
\end{center}

A média corresponde a área hachurada na figura acima

\begin{namedtheorem}[Rugosidade Máxima]
  Maior delta entre pico e vale sucessivo, isto é, mede a maior deterioração na peça no intervalo
\end{namedtheorem}

Qualidade:
\begin{itemize}
	\item Excelente medida para detectar falhas locais.
\end{itemize}

Defeitos:
\begin{itemize}
	\item Não é representativo do perfil inteiro
	\item Equipamento específico para realização da medida
\end{itemize}

\begin{center}
\includegraphics[width=4cm]{C:/Users/patri/Downloads/Poli/lateco/carrinhos_teoria/images/carrinhos2_rugo_y.png}
\end{center}

\begin{namedtheorem}[Desvio Aritmético Quadrático(RMS)]
  \[Rq = \sqrt{\frac{y_1^2 + y_2^2 + \dots + y_n^2}{n}}\]
\end{namedtheorem}

\begin{namedtheorem}[Rugosidade Total]
  Maior amplitude entre pico e vale de um perfil da peça inteira no intervalo avaliado.
\end{namedtheorem}

$$l_m = n\times l_e$$

\begin{center}
	\includegraphics[width=4cm]{C:/Users/patri/Downloads/Poli/lateco/carrinhos_teoria/images/carrinhos2_rugo_t.png}
\end{center}

Para informações sobre notação, checar o apendice.
